% !TeX root = resume.tex
\documentclass{resume} % custom class file
\usepackage{resume} % custom styling

%% PREAMBLE %%
\usepackage[english]{babel}

%% SET DEFAULTS %%
% \renewcommand{\defaultseparator}{$|$} % ex. set the default separator

%% END PREAMBLE %%

%% BEGIN DOCUMENT %%
\begin{document}

%% RESUME HEADER %%
\resumeheader[
    alignment=l,
    firstname=Nathalie,
    lastname=Redick,
    % location={Davis, California}
]{
    \inlinelocation{Davis, CA}\separator
    \resumeheaderlink{tel:+15184104084}{+1 (518) 410-4084}[\faPhone]\separator
    \resumeheaderlink{mailto:nrredick@ucdavis.edu}{nrredick@ucdavis.edu}[\faEnvelope]\separator
    \resumeheaderlink{https://linkedin.com/in/nredick}{nredick}[\faLinkedin]\separator
    \resumeheaderlink{https://github.com/nredick}{nredick}[\faGithub]\separator
    \resumeheaderlink{https://orcid.org/0009-0005-5028-5299}{0009-0005-5028-5299}[\faOrcid]
}

%% EDUCATION %%
\section{Education}

\edu[
    layout=1,
    name={University of California, Davis},
    location={Davis, CA},
    date=\daterange{2024/09/01}{2026/06/07},
    gpa=4.00/4.00,
    degree=MSc Geophysics
]{
    % \begin{resumeitemize}
    %     \item \textit{Coursework}: Regional Synthesis of Geophysical \& Geological Data for Geodynamic Modelling
    % \end{resumeitemize}
}

\edu[
    layout=2,
    name=McGill University,
    location={Montreal, QC},
    date=\daterange{2019/09/01}{2023/08/01},
    gpa=3.75/4.00,
    degree={BA Computer Science}
]{
    \begin{resumeitemize}
        \item Minor in Earth \& Planetary Sciences, Supplementary Minor Concentration in Computer Science (Machine Learning)
        % geology & cs
        % \item \textit{Relevant Coursework}: Mineralogy, Petrology, Geology in the Field, Field School I, Earth Physics, Earth System Modelling, Structural Geology, Volcanology, Algorithms \& Data Structures, Data Science, Linear Algebra I \& II, Probability, Statistics, Applied Machine Learning, Probabilistic Programming, Machine Learning Applied to Climate Change.
        % ALT
        % \item Algorithms \& Data Structures, Data Science, Linear Algebra I \& II, Discrete Math, Probability, Statistics, Applied Machine Learning, Probabilistic Programming, Mineralogy, Petrology, Geology in the Field, Earth Physics, Earth System Modelling, Structural Geology, Volcanology.
        % COMP SCI
        % \item Algorithms \& Data Structures, Data Science, Linear Algebra I \& II, Discrete Math, Probability, Statistics, Applied Machine Learning, Probabilistic Programming.
        % EXTENDED CS
        % \item Intro to Software Systems, Intro to Computer Systems, Programming Languages \& Paradigms, Algorithms \& Data Structures, Data Science, Linear Algebra I \& II, Discrete Math, Probability, Statistics, Applied Machine Learning, Probabilistic Programming.
    \end{resumeitemize}
}

%% EXPERIENCE %%
\section{Experience}

\experience[
    layout=1, % 1 or 2
    title=Technology Analyst,
    company=Morgan Stanley,
    location={Montreal, QC}, % use {} when you need a comma in the field
    date=\daterange{2023/07/31}{2024/08/16},
    % link=https://www.morganstanley.com
]{
    \begin{resumeitemize}
        \item Worked collaboratively to provide agile metrics analysis for internal dev. teams globally, user support, \& documentation.
        \item Utilized DB2 SQL, MongoDB, \& Python to process metrics \& maintain project infrastructure.
    \end{resumeitemize}
}

\experience[
    layout=1,
    title=Data Science Intern,
    company=Esri Canada,
    location={Remote},
    date=\daterange{2022/05/17}{2022/08/05},
    % link=https://www.esri.ca
]{
    \begin{resumeitemize}
        \item Automated a workflow for updating national hydrography data using the Multi-Task Road Extractor \textbf{deep learning} model.
        % \item Gained experience with geospatial data formats such as GeoTIFFs, ESRI Shapefiles, TIFFs, \& Geopackages.
        \item Designed new input image layers \& geomorphological indicators that improved the baseline model accuracy by $\sim$4\%. %by designing new input image layers \& geomorphological indicators. % to \textbf{96.3\% accuracy \& 0.85 MIOU}
    \end{resumeitemize}
}

% \experience[
%     layout=1,
%     title=Project Manager,
%     company=RedQuoin,
%     location={Wilton, NY},
%     date=\daterange{2021/05/01}{2021/08/31},
% ]{
%     \begin{resumeitemize}
%         \item Worked part-time to facilitate communication between the software engineer, business consultant, \& the head of the start-up.
%         \item Conceptualized the design of the software's UI/UX \& visual identity to improve user experience.
%     \end{resumeitemize}
% }

% \experience[
%     layout=1,
%     title=Software Development Consultant,
%     company=Redbud Development,
%     location={Wilton, NY},
%     date=\daterange{2020/01/01}{2021/08/31},
%     % link=https://www.esri.ca
% ]{
%     \begin{resumeitemize}
%         \item Designed \& built a macOS desktop app in Python to process design \& project budget data for clients.
%         \item Implementing the app into the workflow \textbf{reduced proposal creation time by 95\%.}
%     \end{resumeitemize}
% }

% \experience[
%     layout=1,
%     title=Software Engineering Intern,
%     company=Blue Spiral Interactive/Albany IT Group,
%     location={Saratoga Springs, NY},
%     date=\daterange{2019/06/01}{2019/08/31}, % check exact days
%     % link=http://bluespiral.io
% ]{
%     \begin{resumeitemize}
%         \item Improved in-house marketing analysis software by working with a team to build a \textbf{RESTful API} for visualising data.
%         % \item Individually developed a pipeline in Python to standardise 10GB of NYS voter registration data to map on QGIS; map was designed to advise a spatially-informed political campaign strategy.
%         \item Self-taught Python, Git, \& QGIS during the internship. Used parallel computing to \textbf{reduce execution time by 97\%}.
%     \end{resumeitemize}
% }

% \experience[
%     layout=1,
%     title=Software Development Intern,
%     company={Garnet River, LLC},
%     location={Saratoga Springs, NY},
%     date=\daterange{2019/02/01}{2019/06/01},
%     % link=https://www.example.com
% ]{
%     \begin{resumeitemize}
%         \item Evaluated the efficacy \& usability of computer vision products from Microsoft, Google, \& AWS.
%     \end{resumeitemize}
% }

% \experience[
%     layout=1,
%     title=Busser,
%     company=DZ Restaurants,
%     location={Saratoga Springs, NY},
%     date=\daterange{2018/05/01}{2018/08/31},
%     % link=https://www.example.com
% ]{}

% \experience[
%     layout=1,
%     title=Lifeguard,
%     company=Saratoga Regional YMCA,
%     location={Saratoga Springs, NY},
%     date=\daterange{2017/05/01}{2018/08/31},
%     % link=https://www.example.com
% ]{}

%% RESEARCH %%
\section{Research}

\experience[
    layout=1,
    title=Machine Learning For Geospatial Analysis,
    company=McGill University,
    location={Montreal, QC},
    date=\daterange{2022/09/01}{2024/07/01},
    link=https://www.example.com
]{
    % Advised by Dr.~James Kirkpatrick \& Dr.~Matthew Tarling.
    \begin{resumeitemize}
        \item Designed a guided machine learning workflow for geospatial analysis.
        % \item Our objective was to create a tool that can be used by anyone, regardless of their technical background.
    \end{resumeitemize}
}

\experience[
    layout=1,
    title=Using U-Net to Identify Landslides,
    company=McGill University,
    location={Montreal, QC},
    date=\daterange{2021/05/01}{2022/08/31},
    link=https://www.example.com
]{
    % Advised by Dr.~James Kirkpatrick \& Dr.~Veronica Prush.
    \begin{resumeitemize}
        \item Implemented an image segmentation ML model to identify landslides using geophysical \& morphological indicators.
        % \item Current iteration of the model boasts \textbf{95.3\% accuracy \& a loss of 0.19.}
        % \item Currently \textbf{collaborating with the California Geological Survey} to expand the project scope.} %implement new methods into the landslide identification workflow \& to improve the performance of the model.
    \end{resumeitemize}
}

% \experience[
%     layout=1,
%     title=Undergraduate Research Assistant,
%     company={Earthquake Processes Research Group, McGill University},
%     location={Montreal, QC},
%     date=\daterange{2021/01/01}{2021/08/31},
%     % link=https://www.eps.mcgill.ca/~crowe/EQprocesses
% ]{
%     \begin{resumeitemize}
%         \item Individually designed \& built a website using \textbf{HTML/CSS \& JS} to communicate seismological data of Qu\'ebec to promote public awareness about local earthquake hazards.
%     \end{resumeitemize}
% }

\section{Field Work}

\experience[
    layout=1,
    title=Graduate Volcanology Seminar,
    company=McGill University,
    location={Montreal, QC},
    date=October 2022,
]{
    \begin{resumeitemize}
        \item Participated in a 1-week field seminar to study the volcanological features \& history of the Long Valley caldera in California.
    \end{resumeitemize}
}

\experience[
    layout=1,
    title=Field School I (2.5 weeks),
    company=McGill University,
    location={Montreal, QC},
    date=\daterange{2021/05/01}{2021/05/17},
]{
    \begin{resumeitemize}
        \item Produced maps of geologic units \& structures in both Rainbow Basin, CA \& Dublin Gulch, CA.
        % \item Mapped soil ??? in the High Sierra Nevada Mountains.
        \item Gained experience with field mapping, using a Brunton compass, \& topographic maps.
    \end{resumeitemize}
}

% todo: add projects section

%% AWARDS %%
\section{Awards}

% \award[
%     layout=1,
%     title=Bogo Hack,
%     organization=MAIS Hacks 2022,
%     date=\daterange{2022/10/01}{2022/10/02},
% ]{
%     \begin{resumeitemize}
%         \item Created a \href{https://devpost.com/software/read-between-the-wines}{web app} using Python \& Google Earth Engine that predicts the price \& quality of a bottle of wine based on climatological conditions of the year \& region it was produced.
%     \end{resumeitemize}
% }

% \award[
%     layout=1,
%     title={Best Design \& Most Fun; Most Creative Game Dev Hack},
%     organization=McHacks9,
%     % date=\daterange{2022/01/21}{2022/01/23},
%     date=2022, % todo fix date formatting
% ]{
%     \begin{resumeitemize}
%         \item Competed against 331 participants \& created \href{https://pacdemic-man-2vsjz.ondigitalocean.app/}{Pan(demic)-Man}, a COVID-19-themed Pac-Man built with \textit{Unity Game Engine} \& C\#.
%     \end{resumeitemize}
% }

% \award[
%     layout=1,
%     title={Best AI Hack for Art},
%     organization=MAIS Hacks 2021,
%     date=\daterange{2021/10/02}{2021/10/03},
% ]{
%     \begin{resumeitemize}
%         \item Competed against 111 participants \& created \href{https://devpost.com/software/maispeare}{MAISpeare}, a LSTM-driven web app (\textit{Python, HTML/CSS}) that generates a poem from any image.
%     \end{resumeitemize}
% }

\award[
    layout=1,
    title={Geotop 2021 Scholarship Competition},
    organization=Geotop,
    date=2021, % todo check exact date
    amount=\$1500
]{
    \begin{resumeitemize}
        \item Selected based on my research proposal to \textit{Use ML to Identify Landslides} \& academic performance.
    \end{resumeitemize}
}

% \award[
%     layout=1,
%     title={Best Overall Hack},
%     organization=MAIS Hacks 2020,
%     date=\daterange{2020/10/03}{2020/10/04},
% ]{
%     \begin{resumeitemize}
%         \item Lead a team against 115 participants to create a XGBoost-driven \href{https://devpost.com/software/mbti-personality-classifier-2eho6w}{web app} to predict MBTI Personality Type based on Twitter data.
%     \end{resumeitemize}
% }

\award[
    layout=1,
    title={Alma Mater Scholarship},
    organization=McGill University,
    date=2019, % todo check exact date
    amount=\$3000
]{
    % \begin{resumeitemize}
    %     \item Entrance bursary to McGill University for academic excellence.
    % \end{resumeitemize}
}

% \award[
%     layout=1,
%     title={Stat Staff Professionals Computer Science Scholarship},
%     organization=Saratoga Springs High School,
%     date=2019, % todo check exact date
%     amount=\$1000
% ]{
%     \begin{resumeitemize}
%         \item Selected amongst $\sim40$ students for academic excellence \& demonstrated potential in computer science.
%     \end{resumeitemize}
% }

% todo: finish adding old hs awards

%% EXTRA-CURRICULARS %%

% todo add geophysics reading group
\section{Extra-curriculars}

\experience[
    layout=1,
    title=AWG Student Mentor,
    company=Association of Women Geoscientists (AWG) at UC Davis,
    location={Davis, CA},
    date=January 2024--Present, % jan 06 2024
]{
    \begin{resumeitemize}
        % \item Mentor for the AWG 1-on-1 mentorship program between graduate students \& undergraduates at UC Davis.
        \item Assist a student in learning new skills, building job applications \& resumes; discussing the science field \& graduate school.
        \item Guide the student in developing an exploratory research project in the geosciences.
    \end{resumeitemize}
}

\experience[
    layout=1,
    title=Datalab Affiliate,
    company=UC Davis Datalab,
    location={Davis, CA},
    date= October 2024--Present, %\daterange{2024/10/21}{2023/04/31},
]{
    \begin{resumeitemize}
        \item Participate \& assist in workshops related to data science \& computational pedagogy.
        \item Help maintain the affiliated KeckCAVE Virtual Reality research lab in the Earth \& Planetary Sciences department.
    \end{resumeitemize}
}

\experience[
    layout=1,
    title=Vice President of Communications,
    company=The Monteregian Society,
    location={Montreal, QC},
    date=\daterange{2020/09/01}{2023/04/31},
]{
    \begin{resumeitemize}
        \item Managed communications for the undergraduate student council for Earth \& Planetary Sciences at McGill University.
        % \item Designed \& built the council's website to host student resources, events, \& other information.
    \end{resumeitemize}
}

% \experience[
%     layout=1,
%     title=Member,
%     company={ML for Geoscience Reading Group, McGill University},
%     location={Montreal, QC},
%     date=\daterange{2021/01/01}{2021/05/31},
% ]{
%     \begin{resumeitemize}
%         \item Participated in a reading group to examine current papers in ML applications in the geosciences.
%     \end{resumeitemize}
% }

% todo NHS, SWAM

%% PROFESSIONAL DEVELOPMENT %%
\section{Professional Development}

\experience[
    layout=1,
    title=Instructor Training: Introduction to Computational Pedagogy,
    company=UC Davis Datalab,
    location={Davis, CA},
    date=December 2024,
]{
    \begin{resumeitemize}
        \item Two-day workshop on evidence-based teaching, inclusive pedagogy, and instructional design for computational skills.
        \item Strategies for teaching students from non-computational backgrounds, designing inclusive learning environments, and adapting to in-person/virtual/hybrid formats.
    \end{resumeitemize}
}

\experience[
    layout=1,
    title=SCIWS12 Tutorial on Machine Learning \& Deep Learning,
    company=American Geoscience Union,
    location={Virtual},
    date=December 2020,
    link=https://www.agu.org/Events/FM20/SCIWS12-15-Tuesday-December
]{
    \begin{resumeitemize}
        \item Attended a full-day technical workshop on machine learning \& deep learning for the environmental \& geosciences.
    \end{resumeitemize}
}

\experience[
    layout=1,
    title=Accelerated Introduction to ML,
    company=McGill Artificial Intelligence Society,
    location={Montreal, QC},
    date=\daterange{2020/01/14}{2020/04/09},
    link=https://mcgillai.com/mais202
]{
    \begin{resumeitemize}
        \item Selected through a technical interview to participate in a \textbf{10-week} accelerated course of ML. % through weekly lectures, assignments, \& a final project fair
        % \item Webscraped data to train a CNN to classify rock/mineral/fossil sample images into 4 classes; deployed as a webapp.
    \end{resumeitemize}
}


%% SKILLS %%
\section{Skills}

\skills[
    cols=1, %
]{
    \textbf{Programming Languages}: Python, Julia, C{\small++}, C, Java, DB2/SQL/MySQL, R, Bash, MATLAB, HTML/CSS, OCaml, MIPS Assembly\\
    \textbf{Tools}: Git, Linux/Unix, \LaTeX, Jupyter, QGIS/ArcGIS, AWS EC2, VS Code, RESTful APIs, MongoDB, Jira, Jenkins, Liquibase
}

%% PUBLICATIONS \& PRESENTATIONS %%
\section{Publications \& Presentations}
\nocite{*}
\printbibliography[heading=none]

% \section{Interests}

%% FOOTER %%
\lastupdated

%% END DOCUMENT %%
\end{document}
